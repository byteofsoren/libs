\section{Introduction}%
\label{sec:introduction}
In this report the construction of a skeleton character device is discussed briefly.
The skeleton driver began its live as the name short print, constructed to connect a parallel port to the Linux system.
Building a less complicated character device is done by modifying the driver so it don't contain any unnecessary code.




% ----------------------------- Method  -------------------------------------
\section{Method}%
\label{sec:method}
In this case all the software for all input output \(IO\) and all the queues are striped out of the system.
Each of the drivers different part is briefly discussed in each its own sub subsection.
\subsubsection{Initialization of the driver}%
\label{ssub:method:init}
In the initialisation of the driver a  change that unfortunately made the system more complex to understand is that it was choice to include the new system to register character device on the kernel present from kernel $2.5$ and onward.
But to be able to provide backward comparability the old system is present before compiling by uncomment the line \verb|#define REGCHDEV|.
The new system also needs a new \verb|structs| and a new variable to register the device on the kernel.
Those are also defined with in a \verb|#ifndef REGCHDEV|  so they only are present on the new system.
At the end of the skeleton initialisation step there are an open aria where the next developer could allocate necessary resources when writing an new driver for the Linux kernel.

\subsubsection{Write to the driver}%
\label{ssub:method:write}
Writing information to the driver form the user space, calls the write function in the driver.
In this implementation this is only a simple to understand buffer with an example that writes a message to the kernel log.
Note that the example is commented out with an \verb|/*example => .. */| and an empty block were custom code could be implemented.

\subsubsection{Reading the driver}%
\label{ssub:method:read}
The reversed way is done by comping data to the user space and then after that it increases the file position of the pointer withe the amount it copied. There're also a blank aria with in the driver to write the necessary code.


\subsubsection{Cleanup process}%
\label{ssub:method:cleanup}
The different methods for allocating the character device at the kernel also affects how the system removes the driver when removed for the kernel. This is also done by using the previously discussed define \verb|REGCDEV|. There are also a free block at the end to free previously allocated resources.



% ----------------------------- Results -------------------------------------
\section{Results}%
\label{sec:results}
Compiling and \verb|insmod| the driver results in a bit of text getting pushed on the kernel log that tells the user that the driver was successfully registered at the kernel.
\lstinputlisting[style=text]{logs/dmsg.log}

\section{Discussion}%
\label{sec:Discussion}
The provided solution on the new \verb|structs| and variables to use the new system to register the character device on the kernel could be cleaned up allot but not with out massive rewrite not required by from the instructions.




